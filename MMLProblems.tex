\section{Translations via MathML}\label{sec:trans-mml}
The problems to use MathML data as an intermediate format can be demonstrated by using LaTeXML, \Maple, and \Mathematica{} to generate content MathML from their inputs. While the inputs for \gls*{cas} are unique, ideally all systems should produce the same content MathML data. Also, for semantic \LaTeX, LaTeXML should be able to generate the correct content MathML data. As an example we generated content MathML via LaTeXML for
\begin{equation}\label{eq:jac-lml}
\verb|\JacobiP{\alpha}{\beta}{n}@{\cos@{a\Theta}}|,
\end{equation}
and via \Maple{} and \Mathematica{} for the manually verified appropriated  translations of (\ref{eq:jac-lml})
\begin{eqnarray}
&& \verb|JacobiP(n,alpha,beta,cos(a*Theta))|,\text{ and}\label{eq:jac-map}\\
&& \verb|JacobiP[n,\[Alpha],\[Beta],Cos[a \[CapitalTheta]]]|.\label{eq:jac-mat}
\end{eqnarray}
Listings~\ref{lst:latexml}-~\ref{lst:mathematica} shows the generated content MathML. As the listings show, all generated MathML distinguish to each other. The problems are not only flaws in their export implementations but also in the fact that even content MathML is not unique. Based on this fact and the differences in the generated content MathML between each system, we decided to not use MathML as an intermediate format.

\begin{lstlisting}[label={lst:latexml},mathescape=true,caption=The generated content MathML for~(\ref{eq:jac-lml}) using LaTeXML with the command \texttt{latexmlc} and the set of semantic macros.]
<math xmlns="http://www.w3.org/1998/Math/MathML" $\ldots$>
  <semantics>
    <apply>
      <apply>
        <csymbol cd="dlmf">Jacobi-polynomial-P</csymbol>
        <ci>$\alpha$</ci>
        <ci>$\beta$</ci>
        <ci>$n$</ci>
      </apply>
      <ci>$x$</ci>
    </apply>
  </semantics>
</math>
\end{lstlisting}

\vspace{0.4cm}
\begin{lstlisting}[label={lst:maple},mathescape=true,caption=The content MathML for~(\ref{eq:jac-map}) generated by Maple 16 (including manually added linebreaks). We were using the \texttt{MathML[ExportContent]} command to produce the output.]
<math xmlns='http://www.w3.org/1998/Math/MathML'>
  <apply id='id6'>
    <csymbol id='id1' 
      definitionURL='http://www.maplesoft.com/MathML/JacobiP'>
        JacobiP
    </csymbol>
    <ci id='id2'>n</ci>
    <ci id='id3'>alpha</ci>
    <ci id='id4'>beta</ci>
    <ci id='id5'>x</ci>
  </apply>
</math>
\end{lstlisting}

\vspace{0.4cm}
\begin{lstlisting}[label={lst:mathematica},mathescape=true,caption=The content MathML for~(\ref{eq:jac-mat}) generated by Mathematica. We were using the \texttt{ExportString} command to produce the output.]
<math xmlns='http://www.w3.org/1998/Math/MathML'>
 <apply>
  <ci>JacobiP</ci>
  <ci>n</ci>
  <ci>&#945;</ci>
  <ci>&#946;</ci>
  <ci>x</ci>
 </apply>
</math>
\end{lstlisting}
\end{document}