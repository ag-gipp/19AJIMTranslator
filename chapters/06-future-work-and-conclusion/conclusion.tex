The results and conclusions of the translations of the formula set from the \DLMF{} are diverse. Some test cases have shown us that the set of semantic macros is not perfect yet and needs to be improved, seen in the example of the macro definition for the Wronskian symbol. However, simply adding more and more macros to improve the set is not the right decision. In that case we would end up in something similar to \sTeX, which is overloaded and too complicated to use. Obviously, if the system is no longer handy, it will not be used.

On the other hand, the concept of the translator has been proven. For example, one test case has revealed an overall sign error in an equation in the \DLMF~\cite[(14.5.14)]{NIST:DLMF} and therefore the validation system has shown the potential to become a strong verification tool for mathematical compendia. The test cases have also shown how difficult it is to validate a translated expression and have uncovered the problems of translations between two systems with different sets of supported functions. Our validation techniques also assume the correctness of the simplification and computational algorithms in \gls{cas}. However, combining those techniques and automatically running translation checks can potentially not only discover errors in mathematical compendia, but also detect errors in simplifications or computations in \gls{cas}.

Finally we can conclude that the project has been successful and the translator has great potential to become a handy translation tool. Nonetheless, it is without any doubts an ever-growing project and needs to be improved constantly.