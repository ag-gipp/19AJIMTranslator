The results and conclusions of the translations of the formula set from the \DLMF{} are diverse. Some test cases have shown us that the set of semantic macros is not perfect yet and needs to be improved, seen in the example of the macro definition for the Wronskian symbol. However, simply adding more and more macros to improve the set is not the right decision. In that case, we would end up in something similar to \sTeX, which is overloaded and too complicated to use. If the system is no longer handy, it will not be used.

On the other hand, the concept of the translator has been proven to discovering errors in the online compendia \gls*{dlmf}. The test cases have also shown how difficult it is to validate a translated expression and have uncovered the problems of translations between two systems with different sets of supported functions. Our validation techniques also assume the correctness of the simplification and computational algorithms in \gls*{cas}. However, combining those techniques and automatically running translation checks can potentially not only discover errors in mathematical compendia but also detect errors in simplifications or computations in \gls*{cas}.

The tasks for future work are diverse. The main task to improve the translator by implementing more functions and features. For example, for the current state, only translations to \Maple's standard function library were implemented. \Maple{} allows to load extra packages dynamically and therefore support several more functions. This feature would drastically increase the number of possible translations. With such improvements, further work on evaluation techniques become worthwhile to evaluate \gls*{dlmf} and \gls*{cas}.

Furthermore, the translator was designed to be easily extendable. This allows to implement translations for other \gls*{cas} without much effort.

The biggest weakness of the translator is still the semantic macros. Currently, we are working on mathematical information retrieval techniques to allow an extension for the translator for generic \LaTeX{} inputs.