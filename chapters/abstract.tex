\begin{abstract}
\glsresetall
\noindent
This paper presents a translation tool between representations of semantically enriched mathematical formulae in a \gls*{wp} and its corresponding representations in a \gls*{cas}. The representatives for the \gls*{cas} are \Maple{} and \Mathematica{}, while the representative \gls*{wp} is \LaTeX.
%
The \gls*{nist} in the U.S.~has developed a set of semantic \LaTeX{} macros for \gls*{opsf}. The \gls*{dlmf} use these semantic macros to provide a semantically enhanced mathematical online compendium. This kind of semantic macros provides exclusive access to the semantic information of the functions. However, even if the semantics of a representation of one formula is unique, it does not need to match the semantics in another representation for the same formula. While some distinctions are obvious, such as syntactical characteristics, others are more difficult to examine, such as differences in definitions, which leads to error-prone manual translations.
This paper presents an automatic translation system and discusses the problems and suggests solutions. 
%
Furthermore, this paper introduces new evaluation approaches for the developed translation tool. 
With the help of the automatic translation tool, the evaluation experiments were able to discover errors in the \gls*{dlmf} and in the \gls*{cas} Maple.
\end{abstract}