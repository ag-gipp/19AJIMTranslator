\section{Translator Definitions}\label{sec:definition}
To explain translations mathematically, we need to define some techniques first. Mainly, a translation process is a function that maps an input expression to an output expression. The input and output expressions are written in two different formal languages. For a translation, the exact types of input and output language are insignificant. Therefore, we consider mathematical \LaTeX, semantic \LaTeX, \Maple's \texttt{1D} input and \Mathematica{} input as formal languages. 

\begin{definition}[Translation]
Let $\mathcal{L}$ be a set of formal languages and $L_1, L_2 \in \mathcal{L}$ be two formal languages. A \textbf{translation} $\mathrm{tr}$ is a mapping from an input language $L_1$ to an output language $L_2$, denoted as
\begin{equation*}
\mathrm{tr}^{L_1}_{L_2}: L_1 \rightarrow L_2.
\end{equation*}
\end{definition}

\begin{definition}[Forward \& Backward Translation]
Let $\mathcal{L}$ be a set of formal languages. Let $L_1 \in \mathcal{L}$ be the language of semantic \LaTeX{} and $L_2 \in \mathcal{L}$ the language of any \gls{cas}. The function $\mathrm{tr}^{L_1}_{L_2}$ is called \textbf{forward translation to $L_2$} and $\mathrm{tr}^{L_2}_{L_1}$ is called \textbf{backward translation from $L_2$} respectively.
\end{definition}

Therefore, we define $\langTex$ as the formal language of semantic \LaTeX, $\langMaple$ as the formal language of \Maple{} and $\langMathe$ as the formal language of \Mathematica. Since we will only translate expressions between $\langTex$ and a \gls{cas}, we will use a more convenient notation for forward and backward translations. Let $t \in \langTex$ be an arbitrary expression in semantic \LaTeX{}. A forward translation to \Maple{} is denoted as
\begin{equation}
t \overset{\langMaple}{\mapsto} c, \quad \text{with } c \in \langMaple,
\end{equation}
and a backward translation from \Maple{} as
\begin{equation}
t \overset{\langMaple}{\mapsfrom} c, \quad \text{with } c \in \langMaple.
\end{equation}

Note that there is an identity translation between two formal languages $L_1, L_2$ defined if there is a word $w \in L_1 \cap L_2$. We specially denote the identity translation with 
\begin{align}
&\mathrm{id}^{L_1}_{L_2}:\ L_1 \rightarrow L_2,\notag\\
&\mathrm{id}^{L_1}_{L_2}(w) := w,\quad \text{with } w \in L_1 \cap L_2.
\end{align}

In this thesis, we will translate expressions from one system to another. There is no formal definition to compare mathematical expressions in two different systems. As an example, consider the trigonometric cosine function $\cos@{z}$. There is a formal definition in the \gls{dlmf} given~\cite[(4.14.2)]{NIST:DLMF}
\begin{equation}
\cos@{z} = \frac{\expe^{\iunit z} + \expe^{-\iunit z}}{2}.
\end{equation}
The internal definition of the cosine function in \Maple{} is given by
\begin{equation}
\cos@{z} = \frac{1}{2}\expe^{\iunit z} + \frac{1}{2\expe^{\iunit z}}.
\end{equation}
Technically, we would agree that both definitions are equivalent. However, \Maple{} is limited by the possibilities of a computer system. For example, the mathematical constant $\expe$ is irrational and can therefore never be \textit{absolutely equivalent} to the abstract mathematical construct of Euler's number. Instead of defining a new equivalence term for our purpose, we conveniently look for an \textbf{appropriate} translations between two formal languages. Obviously, this is not a formal definition, but should prevent us from translating one mathematical object\footnote{We are aware that it could be even more difficult to formally define what a \textit{mathematical object} is. We consider a mathematical object to be a function or symbol with a formal mathematical definition behind it.} into a completely different object in another language.

Therefore, we would label a translation such as
\begin{equation}\label{eq:cos-def}
\verb|\cos@{z}| \overset{\langMaple}{\mapsto} \verb|cos(z)|
\end{equation}
as \textit{appropriate}, while a translation such as
\begin{equation}
\verb|\cos@{z}| \overset{\langMaple}{\mapsto} \verb|sin(z)|
\end{equation}
would be \textit{inappropriate}. However, note that it is not always as easy as in this example to decide if a translation is appropriate or not. Therefore, we developed validation techniques to automatically check if a translation can be considered appropriate. In addition to this terminology, we introduce \textit{direct translations}. Later in the thesis, we will explain that a translation from one specific mathematical object to its \textit{appropriate} counterpart in the other system is not possible. We call a translation to the \textit{appropriate} counterpart \textbf{direct}. For example, the translation~(\ref{eq:cos-def}) is \textit{direct}, while a translation to the definition of the cosine function
\begin{equation}
\verb|\cos@{z}| \overset{\langMaple}{\mapsto} \verb|(exp(I*z)+exp(-I*z))/2|
\end{equation}
is not a \textit{direct} translation.