Before we start to explain the translation process, we need to introduce some basic information. We will see that it is not as easy as it seems to explain what semantic information is, that there can be hidden complexity even behind simple mathematical expressions, and why we need grammar to understand mathematics. All this requires a comprehensive introduction.

In this chapter we will lay the foundation to understand the complexity and our solutions in this project. The first section~\ref{sec:related-work} talks about the ideas and problems of related work. In the following section~\ref{sec:math-background}, we will give an introduction to complex analysis, branch cuts and special functions in general. Sections~\ref{sec:dlmf} and \ref{sec:semantics} will introduce the \gls{dlmf} and semantic \LaTeX, which is the most important part for our translations to \gls{cas}. Section~\ref{sec:cas} will give a brief introduction to \gls{cas} and mainly discusses \Maple{} and \Mathematica. The \gls{mlp} is another important part of the translation process, because it gives us the opportunity to parse mathematical expressions. Section~\ref{sec:mlp} explains how this works and why this is so important for a translator. This chapter finishes then with some extra definitions for the rest of the thesis.

Later on we will focus on the translation process between mathematical expressions in semantic \LaTeX{} and in \gls{cas}, where the source of semantic \LaTeX{} is the \gls{dlmf} and the source for \gls{cas} expressions is the corresponding \gls{cas}. Because there is no umbrella term for those representations from different sources, we hereafter call them \textit{systems}. A translation between semantic \LaTeX{} and \Maple{} is therefore in general a translation between two different systems.