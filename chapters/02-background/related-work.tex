\section{Related Work}\label{sec:related-work}
Since \LaTeX{} became the de facto standard for writing papers in mathematics, most of the \gls*{cas} provide simple functions to import and export mathematical \LaTeX{} expressions%
\footnote{The selected \gls*{cas} Maple, Mathematica, Matlab, and SageMath provide import and/or export functions for \LaTeX:\\
Maple, \url{http://www.maplesoft.com/support/help/Maple/view.aspx?path=latex} seen 06/2017;\\
Mathematica, \url{https://reference.wolfram.com/language/tutorial/GeneratingAndImportingTeX.html} seen 06/2017;\\
Matlab, \url{https://www.mathworks.com/help/symbolic/latex.html} seen 06/2017;\\
SageMath, \url{http://doc.sagemath.org/html/en/tutorial/latex.html} seen 06/2017}. %
Those tools have two essential problems. They are only able to import simple mathematical expressions, where the semantics are unique. For example, the internal \LaTeX{} macro \verb|\frac| always indicates a fraction. However, for more complex expressions, e.g., the Jacobi polynomial in \cref{tab:JacobiP-usecase}, the import functions fail. The second problem appears in the export tools. Mathematical expressions in \gls*{cas} are fully semantic, otherwise the \gls*{cas} wouldn't be able to compute or evaluate the expressions. During the export process, the semantic information gets lost, because generic \LaTeX{} is not able to carry semantic information. In consequence of these two problems, an exported expression cannot be imported to the same system again in most cases (except those simple expressions described above). Our tool should solve these problems and provide round trip translations between \LaTeX{} and \gls*{cas}.

The semantics must be well known before an expression can be translated. There are two naiv approaches to solve that problem: (1) someone could specify the semantic information during the writing process (pre-defined semantics) or (2) the translator can find out the right semantic information in general mathematical expressions before it translates the expression. So called \textit{interactive documents}\footnote{There is no adequate definition what interactive documents are. However, this name is widely used to describe electronic document formats that allow interactivity to change the content in real time.}, such as the \gls*{cdf}~\parencite{CDF:Wolfram} by Wolfram Research or the \textit{worksheets} by \Maple{}, try to solve this problem with approach (2) and allow to embed semantic information into the input. Those complex document formats require specialized tools to show and work with the documents (Wolfram CDF Player, or \Maple{} for the \textit{worksheets}). The \JOBAD{} architecture~\parencite{JOBAD:orig} is able to create web based interactive documents and uses \gls*{omdoc}~\parencite{OMDoc} to carry semantics. The documents can be viewed and edited in the browser. Those \JOBAD{-documents} also allow to perform computations during \gls*{cas}. This gives the opportunity to calculate, compute and change mathematical expressions directly in the document. The translation performs in the background, invisible for the user. Similar to the \JOBAD{} architecture other interactive web documents exist, such as \textit{MathDox}~\parencite{MathDox} and \textit{The Planetary System}~\parencite{Planetary}. All of them demonstrate the potential for the education system.

%All of these systems carry semantic information. Most of them create their own way to make this possible. The web based documents mostly use standardized ways, such as the content \gls*{mathML}, a specialization of \gls*{mathML}~\parencite{MathML}, that focuses on the semantics of an expression rather than any particular rendering for the expression. Since \LaTeX{} is the de facto standard to write mathematical expressions in documents there are projects to extend \LaTeX{} in a way that enable to carry the semantic information directly in \LaTeX. There are two known projects that created a semantic version of \LaTeX: \sTeX{}~\parencite{sTeX} developed by Michael Kohlhase and the \Macro s developed by Bruce Miller. Our translator uses the \Macro s rather than \sTeX. A full explanation of this decision will be given in \cref{sec:semantics}.

%However, \textit{interactive documents}, web based or not, have an essential problem: all of them create an entire new document format. We want to create an independent lite translation tool for \LaTeX. Once we achieve this goal, an integration into \textit{interactive documents} might be possible, besides many other possible applications.

Another approach tries to avoid translation problems by allow computations directly via the \LaTeX{} compiler, e.g., \textit{LaTeXCalc}~\parencite{LatexCalc}. Those packages are limited to the possibilities of the compiler and therefore not as powerful as \gls*{cas}. A workaround for this case is \textit{sagetex}~\parencite{Sagetex}, which is a \LaTeX{} package interface for the open source \gls*{cas} \textit{sage}\footnote{An abbreviation for \textit{SageMath}.}. This package allows \textit{sage} commands in \TeX{-}files and uses \textit{sage} in the background to compute the commands. In this scenario, a writer still needs to manually translate expressions to the syntax of \textit{sage}.

%Obviously, the first example is not as powerful as a \gls*{cas} would be. However, \textit{sagetex} does not really solve our problem of the scientific workflow. The writer still needs to translate expressions manually. The only difference is that the author can write computations directly into the \TeX -file. However, these approaches have the potential for a powerful combination with our translation tool.

%In summary, today's translation tools are either specialized for one specific \gls*{cas} (such as the import and export functions of the \gls*{cas}) or they completely avoid the translation process by creating a new document format (such as \textit{interactive documents}). Our goal is to provide a lite translation tool for mathematical \LaTeX{} expressions and multiple \gls*{cas}.

There exists two approaches that allow to embed semantic information within \LaTeX{} expressions by using custom macros. Namely, \sTeX{}~\parencite{sTeX} developed by Kohlhase and the \Macro s developed by Miller~\parencite{DLMF:Macros}. This paper shows that it is possible to develop a context-free translation tool using the semantic macros introduced by these two projects. \sTeX{} aims to embed semantic information for general mathematics with a comprehensive set of macros. The macro set developed by Miller introduces new macros for special functions, orthogonal polynomials, and mathematical constants. Each of these macros ties specific character sequences to a well-defined mathematical object and is linked with the corresponding definition in the \gls*{dlmf} or \gls*{drmf}. Therefore, we call these semantic macros \Macro s. These semantic macros are internally used in the \gls*{dlmf} and the \gls*{drmf}. Because of the linked definitions to the \gls*{dlmf} and the special macros for functions and mathematical constants this paper using the \Macro s for performing translations to \gls*{cas} instead of using \sTeX.