\section{Related Work}\label{sec:related-work}
Since \LaTeX{} became the de facto standard for writing papers in mathematics, most of the \gls{cas} provide simple functions to import and export mathematical \LaTeX{} expressions%
\footnote{The selected \gls{cas} Maple, Mathematica, Matlab, and SageMath provide import and/or export functions for \LaTeX:\\
Maple, \url{http://www.maplesoft.com/support/help/Maple/view.aspx?path=latex} seen 06/2017;\\
Mathematica, \url{https://reference.wolfram.com/language/tutorial/GeneratingAndImportingTeX.html} seen 06/2017;\\
Matlab, \url{https://www.mathworks.com/help/symbolic/latex.html} seen 06/2017;\\
SageMath, \url{http://doc.sagemath.org/html/en/tutorial/latex.html} seen 06/2017}. %
Those tools have two essential problems. First of all, they cannot import mathematical expressions where the semantic information is absent, which is the usual case for generic \LaTeX{} expressions. Therefore, they are only able to import simple mathematical expressions, where the semantics are unique. For example, the internal \LaTeX{} macro \verb|\frac| always indicates a fraction. These import tools fail for more complex expressions, for example the Jacobi polynomial in \cref{tab:JacobiP-usecase}. The second problem appears in the export tools. Mathematical expressions in \gls{cas} are fully semantic, otherwise the \gls{cas} wouldn't be able to compute or evaluate the expressions. During the export process, the semantic information gets lost, because generic \LaTeX{} is not able to carry semantic information. In consequence of these two problems, an exported expression cannot be imported to the same system again in most cases (except those simple expressions described above). Our tool should solve these problems and provide round trip translations between \LaTeX{} and \gls{cas}.

Obviously, the semantics must be well known before an expression can be translated. There are two approaches to solve that problem: someone could specify the semantic information during the writing process (pre-define semantics) or the translator can find out the right semantic information in general mathematical expressions before it translates the expression. The second approach is unreasonably difficult compared to the pre-definition approach. We will talk about those difficulties in the following sections. We will also introduce a new approach in \cref{ch:conc-future-work} to achieve the second idea.

However, the pre-definition approach is simple and easy to realize, as long as the system can carry semantic information. A typical, not mathematical example are Wikipedia articles~\parencite{Wiki}. An author of such an article can specify further information about symbols, words and sentences by creating a hyperlink to a more detailed explanation. Usually, the author defines the hyperlink, when he is writing the article or he needs to manually add the hyperlink later on. Similar to this approach are \textit{interactive documents}\footnote{There is no adequate definition what interactive documents are. However, this name is widely used to describe electronic document formats that allow interactivity to change the content in real time.}, such as the \gls{cdf}~\parencite{CDF:Wolfram} by Wolfram Research or the \textit{worksheets} of \Maple{}, which are also sometimes referred to as interactive documents. Those documents pre-define the semantics to allow computations. Those complex document formats require specialized tools to show and work with the documents (Wolfram CDF Player, or \Maple{} for the \textit{worksheets}). The \JOBAD{} architecture~\parencite{JOBAD:orig} is able to create web based interactive documents and uses \gls{omdoc}~\parencite{OMDoc} to carry semantics. The documents can be viewed and edited in the browser. Those \JOBAD{-documents} also allow to perform computations during \gls{cas}. This gives the opportunity to calculate, compute and change mathematical expressions directly in the document. The translation performs in the background, invisible for the user. Similar to the \JOBAD{} architecture other interactive web documents exist, such as \textit{MathDox}~\parencite{MathDox} and \textit{The Planetary System}~\parencite{Planetary}. All of them demonstrate the potential for the education system.

All of these systems carry semantic information. Most of them create their own way to make this possible. The web based documents mostly use standardized ways, such as the content \gls{mathML}, a specialization of \gls{mathML}~\parencite{MathML}, that focuses on the semantics of an expression rather than any particular rendering for the expression. Since \LaTeX{} is the de facto standard to write mathematical expressions in documents there are projects to extend \LaTeX{} in a way that enable to carry the semantic information directly in \LaTeX. There are two known projects that created a semantic version of \LaTeX: \sTeX{}~\parencite{sTeX} developed by Michael Kohlhase and the \Macro s developed by Bruce Miller. Our translator uses the \Macro s rather than \sTeX. A full explanation of this decision will be given in \cref{sec:semantics}.

However, \textit{interactive documents}, web based or not, have an essential problem: all of them create an entire new document format. We want to create an independent lite translation tool for \LaTeX. Once we achieve this goal, an integration into \textit{interactive documents} might be possible, besides many other possible applications.

Another approach that tries to avoid the translation problem are \LaTeX{} packages that allow computations directly through the \LaTeX{} compiler (for example \textit{LaTeXCalc}~\parencite{LatexCalc}) or allow \gls{cas} commands for computations directly in \TeX -files. For example, \textit{sagetex}~\parencite{Sagetex} is an interface \LaTeX{} package for the open source \gls{cas} \textit{sage}\footnote{An abbreviation for \textit{SageMath}.}. This package allows \textit{sage} commands in \TeX -files and uses \textit{sage} in the background to compute the commands. Obviously, the first example is not as powerful as a \gls{cas} would be. However, \textit{sagetex} does not really solve our problem of the scientific workflow. The writer still needs to translate expressions manually. The only difference is that the author can write computations directly into the \TeX -file. However, these approaches have the potential for a powerful combination with our translation tool.

In summary, today's translation tools are either specialized for one specific \gls{cas} (such as the import and export functions of the \gls{cas}) or they completely avoid the translation process by creating a new document format (such as \textit{interactive documents}). Our goal is to provide a lite translation tool for mathematical \LaTeX{} expressions and multiple \gls{cas}.