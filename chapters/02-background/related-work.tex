\section{Related Work}\label{sec:related-work}
Since \LaTeX{} became the de facto standard for writing papers in mathematics, most of the \gls*{cas} provide simple functions to import and export mathematical \LaTeX{} expressions%
\footnote{The selected \gls*{cas} Maple, Mathematica, Matlab, and SageMath provide import and/or export functions for \LaTeX:\\
Maple, \url{http://www.maplesoft.com/support/help/Maple/view.aspx?path=latex} seen 06/2017;\\
Mathematica, \url{https://reference.wolfram.com/language/tutorial/GeneratingAndImportingTeX.html} seen 06/2017;\\
Matlab, \url{https://www.mathworks.com/help/symbolic/latex.html} seen 06/2017;\\
SageMath, \url{http://doc.sagemath.org/html/en/tutorial/latex.html} seen 06/2017}. %
Those tools have two essential problems. They are only able to import simple mathematical expressions, where the semantics are unique. For example, the internal \LaTeX{} macro \verb|\frac| always indicates a fraction. However, for more complex expressions, e.g., the Jacobi polynomial in \Cref{tab:JacobiP-usecase}, the import functions fail. The second problem appears in the export tools. Mathematical expressions in \gls*{cas} are fully semantic. Otherwise the \gls*{cas} wouldn't be able to compute or evaluate the expressions. During the export process, the semantic information gets lost, because generic \LaTeX{} is not able to carry sufficient semantic information. In consequence of these two problems, an exported expression cannot be imported to the same system again in most cases (except for simple expressions such as those described above). Our tool should solve these problems and provide round-trip translations between \LaTeX{} and \gls*{cas}.

The semantics must be well known before an expression can be translated. There are two main approaches to solve that problem: (1) someone could specify the semantic information during the writing process (pre-defined semantics); and (2) the translator can find out the right semantic information in general mathematical expressions before it translates the expression. So-called \textit{interactive documents}\footnote{There is no adequate definition what interactive documents are. However, this name is widely used to describe electronic document formats that allow for interactivity to change the content in real time.}, such as the \gls*{cdf}~\parencite{CDF:Wolfram} by Wolfram Research, or \textit{worksheets} by \Maple{}, try to solve this problem with the approach (2) and allow one to embed semantic information into the input. Those complex document formats require specialized tools to show and work with the documents (Wolfram CDF Player, or \Maple{} for the \textit{worksheets}). The \JOBAD{} architecture~\parencite{JOBAD:orig} is able to create web-based interactive documents and uses \gls*{omdoc}~\parencite{OMDoc} to carry semantics. The documents can be viewed and edited in the browser. Those \JOBAD{-documents} also allow one to perform computations via \gls*{cas}. This gives one the opportunity to calculate, compute and change mathematical expressions directly in the document. The translation performs in the background, invisible to the user. Similar to the \JOBAD{} architecture, other interactive web documents exist, such as \textit{MathDox}~\parencite{MathDox} and \textit{The Planetary System}~\parencite{Planetary}. All of these demonstrate the potential of the educational system.

Another approach tries to avoid translation problems by allowing computations directly via the \LaTeX{} compiler, e.g., \textit{LaTeXCalc}~\parencite{LatexCalc}. Those packages are limited to the abilities of the compiler and therefore are not as powerful as \gls*{cas}. A workaround for this case is \textit{sagetex}~\parencite{Sagetex}, which is a \LaTeX{} package interface for the open source \gls*{cas} \textit{sage}\footnote{An abbreviation for \textit{SageMath}.}. This package allows \textit{sage} commands in \TeX{-}files and uses \textit{sage} in the background to compute the commands. In this scenario, a writer still needs to manually translate expressions to the syntax of \textit{sage}.

There exist two approaches for marking up mathematical \TeX/\LaTeX{} documents semantically with \TeX{} macros. Namely, \sTeX{}~\parencite{sTeX} developed by Kohlhase and the \Macro s developed by Miller~\parencite{DLMF:Macros}. This paper shows that it is possible to develop a context-free translation tool using the semantic macros introduced by these two projects. The goal of \sTeX{} was to markup the functional structure of mathematical documents so that they can be exported to the \gls*{omdoc} format. The macro set developed by Miller introduces new macros for special functions, orthogonal polynomials, and mathematical constants. Each of these macros ties specific character sequences to a well-defined mathematical object and is linked with the corresponding definition in the \gls*{dlmf} or \gls*{drmf}. Therefore, we call these semantic macros \Macro s. These semantic macros are internally used in the \gls*{dlmf} and the \gls*{drmf}. We gave the \Macro{} set the favor for developing the translation engine because it provides \gls*{dlmf} definitions for a comprehensive amount of functions. In contrast, \sTeX{} does not focus on the semantics of functions, is too complex to use, and defines diverse macros for symbols and concepts that \gls*{cas} usually does not support.

%Because of the linked definitions to the \gls*{dlmf} and the special macros for functions and mathematical constants this paper using the \Macro s for performing translations to \gls*{cas} instead of using \sTeX.
