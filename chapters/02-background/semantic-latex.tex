\section{Semantic \& Generic \LaTeX}\label{sec:semantics}
Donald E. Knuth developed the typesetting system \TeX{} in 1977~\cite[559]{DigitalTypo}, because he was unsatisfied with the typography of his book \textit{The Art of Computer Programming}, Volume 2~\cites[5]{DigitalTypo}{Knuth}. Around 1980, Leslie Lamport starts to extend \TeX{} by a set of macros, which became a new version of \TeX. According to his name \textit{Lamport}, this new version was called \LaTeX{~}\cite{LATEX}. Nowadays, \LaTeX{} has become the de facto standard for scientists to write their publications~\cite{LATEX:Standard}. This thesis, for example, is written in \LaTeXe, the current version of \LaTeX.

It is possible to customize and extend \LaTeX, for example through style files, where an editor can define new \LaTeX{} macros, environments and document styles.

\subsection{Mathematical Generic \LaTeX}
\LaTeX{} and \TeX{} enables printing of mathematical formulae in a structure similar to handwritten style. For example, the \LaTeX{} expression~\cite[(4.26.13)]{NIST:DLMF}
\begin{equation}\label{eq:latex-ex}
\verb|\int_{0}^{\infty} \sin(t^2) dt = \frac{1}{2} \sqrt{\frac{\pi}{2}}|
\end{equation}
will be rendered as
\begin{equation}\label{eq:latex-ex-r}
\int_{0}^{\infty} \sin(t^2) dt = \frac{1}{2} \sqrt{\frac{\pi}{2}}.
\end{equation}

We call the syntax of such mathematical expressions in \LaTeX{} \textit{mathematical \LaTeX{}} and hereafter use the terms \textit{\LaTeX{} expressions}, \textit{mathematical \LaTeX{} expressions} and \textit{mathematical \LaTeX{}} interchangeably.

Since \TeX{} and \LaTeX{} are typesetting systems, there are several ways to change the rendering of such mathematical expressions. However, for a \gls{cas} the information about the exact rendering of a mathematical expression is not as important as the meaning of the expression, the semantics.

\subsection{Semantic Information}\label{subsec:semantic-latex}
\textit{Semantics} in general is the study of meanings. Semantic information, for example, can be in words, whole sentences, mathematical expressions, and signs.

Consider the equation~(\ref{eq:latex-ex-r}). The meaning behind this expression is an equation. The left-hand side uses an infinite integral over the trigonometric sine function, which can be defined with~\cite[(4.14.1)]{NIST:DLMF}
\begin{equation}
\sin@{z} := \frac{\expe^{\iunit z}-\expe^{-\iunit z}}{2\iunit}.
\end{equation}
The right-hand side of~(\ref{eq:latex-ex-r}) contains a Greek letter $\cpi$, which is known to indicate the mathematical constant for the ratio of a circle's circumference to its diameter. However, $\pi$ can also indicate the prime-counting function. A human reader might conclude that $\pi$ cannot be the prime-counting function, because there is no variable given. In (\ref{eq:latex-ex-r}), $\cpi$ is much more likely the mathematical constant, also because the left-hand side contains the sine function and the mathematical constant $\cpi$ belonging to trigonometric functions.

As demonstrated, there are a lot of meanings behind~(\ref{eq:latex-ex-r}). Some of them are clear, others can be questionable. For a \gls{cas}, the exact meaning of a given input must be unique, otherwise the \gls{cas} cannot compute the input. Therefore, mathematical expressions in \gls{cas} contain accessible semantic information. This is realized to connect character sequences with unique internal procedures and classes, which represent specific mathematical objects.

Mathematical \LaTeX{} usually only contains information about the rendering of the expression and not about the meaning of the expression. However, there is semantic information in mathematical \LaTeX, but it's often hidden. This was shown in the example discussion of the meaning of equation~(\ref{eq:latex-ex-r}). In the plain text of the Greek letter $\pi$ (\verb|\pi|) is no exact meaning. Or, to be more precisely, there are multiple possible meanings, but there is no way to find the correct meaning. However, together with the left-hand side we can conclude that $\pi$ is the mathematical constant in that case. As demonstrated, there is semantic information in the expression, but a reader needs to analyze the whole expression and make conclusions to find it. Hence, for a computer it is difficult to understand mathematical \LaTeX. To provide a translation from mathematical \LaTeX{} to a \gls{cas}, the semantics must be clear and accessible.

We will introduce in the next subsection a semantic version of mathematical \LaTeX. To distinguish this version with classical mathematical \LaTeX, we call classical \LaTeX \textit{generic \LaTeX}.

Obviously, there must be different levels of semantics in expressions. We say an expression in generic \LaTeX{} such as 
\begin{equation}\label{eq:jac-latex}
\verb|P_n^{(\alpha,\beta)}(\cos(a\Theta))|
\end{equation}
contains less semantic information (more semantics are absent) than an appropriate representation in \Maple{}
\begin{equation}\label{eq:jac-maple}
\verb|JacobiP(n,alpha,beta,cos(a*Theta))|.
\end{equation}
This is because the meaning of (\ref{eq:jac-maple}) is unique and accessible, while the exact meaning of (\ref{eq:jac-latex}) must be concluded from the structure and context of the formula. Unfortunately, there is no formal definition for grading semantics or a \textbf{fully semantic} expression. However, we use this phrase, when there is sufficient semantic information accessible to provide an appropriate translation.

\subsection{DLMF/DRMF \LaTeX{} Macro Set}\label{subsec:macros}
Bruce Miller at \gls{nist} has created a set of semantic \LaTeX{} macros~\cite{DLMF:Macros}. Each macro ties specific character sequences to a well-defined mathematical object and is linked with the corresponding definition in the \gls{dlmf} or \gls{drmf}. Therefore, we call these semantic macros \Macro s. These semantic macros are internally used in the \gls{dlmf}, \gls{drmf} and also in this thesis. They are defined in multiple style files and allow optional parameters and multiple renderings. Table~\ref{tab:macro-def} show how a semantic macro is defined in a style file.

\begin{table}[ht]
\centering
\begin{tabular}{lp{7cm}}
	\hline
	\verb|\defSpecFun{function}| & Define a macro \verb|\function|.\\\hline
	\hspace*{0.5cm}\verb|[nparams]| & Number of parameters of the macro.\\\hline
	\hspace*{0.5cm}\verb|[optparams]| & The number of optional parameters of the macro.\\\hline
	\hspace*{0.5cm}\verb|{format}| & Defines the format of the macro. (How it will be displayed).\\\hline
	\hspace*{0.5cm}\verb|[keys]| & Attributes of the function. For example, \textit{meaning} or \textit{role}.\\\hline
	\hspace*{0.5cm}\verb|{arity}| & Number of arguments. Defines how many \{$\bullet$\} blocks come after the $@$ symbol.\\\hline
	\hspace*{0.5cm}\verb|[argformat]| & Format of arguments.\\\hline
	\hspace*{0.5cm}\verb|[altargformat]| & Alternative formats depends on the number of $@$ symbol.\\\hline
\end{tabular}
\caption{Overview for the definition of a semantic macro.}
\label{tab:macro-def}
\end{table}

The \verb|\defSpecFun| is a defined macro to easily define new macros for special functions. It takes a unique character sequence for the function name. The first optional argument \verb|nparams| defines the number of parameters of the special function. Some special functions, for example the associated Legendre functions, can define optional parameters. The general structure of a semantic macro has two parts to display the formula. The first part defines the display for the function symbol (including parameters) in \verb|{format}|. This format is fixed and cannot be changed. In contrast, the display for the arguments of the function can be controlled by the number of $@$ symbols and is defined in multiple optional \verb|[argformat]| blocks.

Consider the \textit{hypergeometric function}~\cite[(15.2.1)]{NIST:DLMF}. The semantic macro 
\begin{equation}
	\verb|\HypergeoF@{a}{b}{c}{z}| 
\end{equation}
is linked to the definition (15.2.1) in the \gls{dlmf}. The \textit{hypergeometric function} provides three different representations. Table~\ref{tab:hypergeo} gives an overview for all of them.

\begin{table}[ht]
\centering
\begin{tabular}{lc}
	\hline
	Semantic macro & Display\\
	\hline
	\tableRowSpace \verb|\HypergeoF@{a}{b}{c}{z}| & $\HypergeoF@{a}{b}{c}{z}$\\
	\tableRowSpace \verb|\HypergeoF@@{a}{b}{c}{z}| & $F\!\left({\genfrac{}{}{0pt}{}{a, b}{c}}; z \right)$\\
	\tableRowSpace \verb|\HypergeoF@@@{a}{b}{c}{z}| & $\HypergeoF@@@{a}{b}{c}{z}$\\
	\hline
\end{tabular}
\caption{The semantic macro for the \textit{hypergeometric function}~\cite[(15.2.1)]{NIST:DLMF} has three different ways to display the function. The different representations are controlled by the number of $@$ symbols.}
\label{tab:hypergeo}
\end{table}

The display of the function name $F$ is fixed, but the number of $@$ symbols control the display of the arguments. The third representation displays only the variable of the function, but hide the given parameters. Therefore, the semantic \LaTeX{} source in the background provide more semantic information than the displayed version. However, all of them represent the \textit{hypergeometric function} defined in the \gls{dlmf}.

The sets of macros from the \gls{dlmf} and the \gls{drmf} are mostly different. The \gls{drmf} uses the macros from the \gls{dlmf} and developed new macros as well. However, it happens that there are two different macros for the same mathematical object defined. One example is the \textit{Gegenbauer polynomial}\footnote{Also known as the \textit{ultraspherical polynomial}.}
\begin{eqnarray}
	&&\verb|\Ultra{\lambda}{n}@{x}|,\\
	&&\verb|\Ultraspherical{\lambda}{n}@{x}|.
\end{eqnarray}

There are currently 685 semantic macros defined. Note that the number of macros is constantly changing, because the project of semantic macros is still work in progress.

\subsection{Weakness of DLMF/DRMF Macros}\label{subsec:semanticVSmacro}
The set of \Macro s gives the opportunity to semantically enhance mathematical expressions. However, there are mostly macros for \gls{opsf}. It is difficult to foresee if a semantic macro provides sufficient semantic information for an appropriate translation, or, as we formulated above, if a semantic macro is really fully semantic.

One example is the \textit{Wronskian}. For example, for two differentiable functions the \textit{Wronskian} is defined as~\cite[(1.13.4)]{NIST:DLMF}
\begin{equation}
	\Wronskian@{w_1(z), w_2(z)} = w_1(z)w_2'(z) - w_2(z)w_1'(z).
\end{equation}
In semantic \LaTeX{} it is used with
\begin{equation}
	\verb|\Wronskian@{w_1(z), w_2(z)}|.
\end{equation}
It turned out, that for an appropriate translation to a \gls{cas}, we need access to the variables of $w_1(z)$ and $w_2(z)$. Therefore, we improved the macro and replaced it to a new version
\begin{equation}
	\verb|\Wron{z}@{w_1(z)}{w_2(z)}|.
\end{equation}
This new macro is displayed in the same way as the old version was displayed, but it provides more information. 

Another version of semantic \LaTeX{} is \sTeX, developed by Michael Kohlhase~\cite{sTeX}. This version of semantically enhanced \LaTeX{} is more comprehensive for general mathematics. It is not particularly useful for \gls{opsf}, and it doesn't link macros to unique mathematical definitions. Therefore, we did not use \sTeX{} in this project, although it provides more semantic information for simple arithmetic expressions. This part is almost completely missing in the \Macro s in its current state. One example is the question of white spaces. Scientists frequently uses white spaces to indicate multiplications. Most \gls{cas} are familiar with this convention and are able to understand such inputs. However, in \Maple's 1-D input notation, an asterisk is mandatory for multiplications. Therefore, we also added a macro \verb|\idot| (for \textit{invisible dot}), to provide a multiplication symbol which is not displayed in \LaTeX, but semantically enhances the \LaTeX{} expressions.