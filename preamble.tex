% general settings
\usepackage[utf8]{inputenc}
\usepackage[ngerman,english]{babel} % german because of Zusammenfassung
\usepackage[T1]{fontenc} 
\usepackage{hyperref}			% for all hyperlinks (also cites)
\usepackage[acronym, nomain]{glossaries} % generate glossaries (for example acronyms) before txfonts!
\usepackage{amsthm}
\usepackage{amssymb}
\usepackage{parskip}		% space between paragraphs
\usepackage{txfonts}            % times new roman
\usepackage{xcolor}				% more colors
\usepackage{enumitem}			% customizable enumerates/itemizes
\usepackage{algorithm}
\usepackage{algpseudocode} %pseudocode
\usepackage{lipsum} % fill text
\usepackage{tabularx}
\usepackage{stmaryrd}
\usepackage{float}
\usepackage{csquotes}
\usepackage{layout}
\usepackage{textcomp} % for prime symbol command: \textquotesingle
\usepackage{authblk} % affilition via \affil

% geometry for margin control and footer control (fancyhdr)
% [showframe] for debugging
% [outer=3cm,inner=4cm,bindingoffset=6mm]
% [width=150mm,top=25mm,bottom=25mm,bindingoffset=6mm]
\usepackage[%
	a4paper,%
	%bindingoffset=1.5cm,% offset for binding
	%marginparwidth=1cm,
	%width=418pt,
	%inner=3.4cm,
	%showframe % debugging
	]{geometry}
\setlength{\headheight}{14pt}
\usepackage{fancyhdr}
%\pagestyle{fancy} % not in general but after the content table

% setup graphics
\usepackage{graphicx}           % for all images
\usepackage{subfig}             % for subfloats (titlepage images)
\usepackage{wrapfig}			% wrapping text around figures
\graphicspath{ {./images/} }    % set the default path for graphics

% http://mirror.utexas.edu/ctan/macros/latex/contrib/adjustbox/adjustbox.pdf
\usepackage[export]{adjustbox}  % for vertical align of images

% own styles
\usepackage{styles/thesistitle} % define preamble variable subtitle
%\usepackage{styles/DLMF/DLMFmath}
%\usepackage{styles/DLMF/DRMFfcns}

% loading late is better...
\usepackage{cleveref}			% better references, should be loaded last
%\crefname{section}{§}{§§}
%\crefformat{section}{§#2#1#3}

% bibliography
\usepackage[
	backend		= biber,
	style		= authoryear-ibid,
	giveninits	= true, % initials for first names
	natbib		= true,
	url			= false,
	isbn		= false,
	doi			= false,
	eprint		= false,
	maxbibnames	= 99,
	minbibnames	= 3
]{biblatex}

\DeclareNameAlias{sortname}{last-first}

\addbibresource{./bibliography/Bibliography.bib}

% ---------------------- DEFINITIONS ---------------------- %
% define hyperlink colors
\hypersetup{
	%draft		= true,% deleted all links and colors of links! Printing mode
	pdftitle	= {Semantic Preserving Bijective Mappings for Representations of Special Functions in Computer Algebra Systems and Word Processors - Exemplified using the DLMF/DRMF, the Word Processor LaTeX and the Computer Algebra Systems Maple and Mathematica},
	pdfsubject={Master's Thesis in Mathematics, Technische Universität Berlin, July 2017.},
	pdfauthor	= {Andr\'e Greiner-Petter},
	pdfcreator	= {Andr\'e Greiner-Petter},
	pdfkeywords	= {Special Functions, LaTeX, Translation},
    %colorlinks, 	% instead of color borders, color the string
    %linkcolor	= {red!50!black}, %{red!0!black} <- print
    %citecolor	= {blue!50!black},%{blue!0!black} <- print
    %urlcolor	= {blue!70!black}
}

% replace "Contents" by "Table of Contents" in babel-english
\addto\captionsenglish{
  \renewcommand{\contentsname}%
    {Table of Contents}%
}

% allow line breaks in URLs in bibliography
\apptocmd{\UrlBreaks}{\do\f\do\m}{}{}
\setcounter{biburllcpenalty}{9000}% Kleinbuchstaben
\setcounter{biburlucpenalty}{9000}% Großbuchstaben

% --------------- DEFINE HEADER AND FOOTER ---------------- %
% The fancyhdr package lets us add things in the left (L), right (R) and centre (C) of the header or footer and also lets us specify a different arrangement depending on whether its on an odd (O) or even (E) page.
% For instance \fancyhead[RO,LE]{Section \thesection}
%\fancyhead{} %reset
%\fancyhead[LE,RO]{\leftmark}
%\fancyhead[RE,LO]{Section \thechapter.\thesection}
%\fancyfoot{} % reset
%\fancyfoot[LE,RO]{\thepage}

% --------------- ALGORITHM DEFINITIONS ---------------- %
\renewcommand{\algorithmicrequire}{\textbf{Input:}} % rename require to input
\renewcommand{\algorithmicensure}{\textbf{Output:}} % rename ensure to output
\newcommand{\NULL}{\textbf{null}}
\renewcommand{\Return}{\textbf{return}}

% --------------------- NEW COMMANDS ---------------------- %
\newcommand{\DLMF}{DLMF}
\newcommand{\DRMF}{DRMF}
\newcommand{\CAS}{CAS}
\newcommand{\Maple}{Maple}
\newcommand{\Mathematica}{Mathematica}
\newcommand{\Macro}{\DLMF/\DRMF{} \LaTeX{} macro}
\newcommand{\MLP}{MLP}
\newcommand{\JOBAD}{{\tt JOBAD}}

\newcommand{\tbs}{\textbackslash}

\newcommand{\langTex}{\mathfrak{sL}}
\newcommand{\langMaple}{\mathfrak{M}_{aple}}
\newcommand{\langMathe}{\mathfrak{M}_{athematica}}

\newcommand{\inertF}{\texttt{InertForm}}

%\newcommand{\1D}{\texttt{1-D}}

\newcommand{\raw}[1]{\texttt{#1}}
\newcommand{\trans}[3]{\mathrm{tr}_{#2}^{#1}\left(#3\right)}

\newcommand{\sTeX}{{\raisebox{-.5ex}S\kern-.5ex\TeX}}

\newcommand{\todo}[1]{{\color{red}\textbf{#1}}}

\newcommand{\tableRowSpace}{\rule{0pt}{0.9\normalbaselineskip}}

\newcommand{\aSingleQuote}{\hspace{0.03cm}\textquotesingle\hspace{0.03cm}}

\newcommand{\tempEnvCitation}{}
\newcommand{\tempEnvSpacing}{}
\newenvironment{myQuote}[2][-0.4cm]%
{%
	\begin{quote}
	\renewcommand{\tempEnvCitation}{#2}
	\renewcommand{\tempEnvSpacing}{#1}
	\itshape
}%
{%
	\vspace{\tempEnvSpacing}
	\begin{flushright}
		\rule{6cm}{0.4pt}\\[-2pt]
		\tempEnvCitation{}
	\end{flushright}%
	\end{quote}
}

\newcommand*{\myRuleTextFill}[2]{%
  \makebox[#1]{%
    \leaders\hrule height \dimexpr 8.2pt\relax depth \dimexpr -7pt\relax \hfill% Left rule
    \enskip{#2}\enskip% Text (and surrounding spaces)
    \leaders\hrule height \dimexpr 8.2pt\relax depth \dimexpr -7pt\relax \hfill\kern0pt}% Right rule
}

\def\myRuleFill{\leavevmode\leaders\hrule height \dimexpr 8.2pt\relax depth \dimexpr -7pt\hfill\kern0pt}

\newtheoremstyle{defTheoStyle}
	{6pt} % Space above
	{2pt} % Space below
	{\itshape} % Body font
	{} % Indent amount
	{\bfseries} % Theorem head font
	{} % Punctuation after theorem head
	{\newline} % Space after theorem head
	{\thmname{#1} \thmnumber{#2}: {\normalfont\textit{(#3)}}} % Threaom head spec (can be left empty, meaning 'normal')
	
\newtheoremstyle{defExampStyle}
	{6pt} % Space above
	{2pt} % Space below
	{\itshape} % Body font
	{} % Indent amount
	{\bfseries} % Theorem head font
	{} % Punctuation after theorem head
	{\newline} % Space after theorem head
	{\thmname{#1} \thmnumber{#2}:} % Threaom head spec (can be left empty, meaning 'normal')

\theoremstyle{defTheoStyle}
\newtheorem{definition}{Definition}[section]
\newtheorem{theorem}{Theorem}[section]
\theoremstyle{defExampStyle}
\newtheorem{example}{Example}[section]

% do not vertically extend pages
\raggedbottom

% -------------------- DEFINE ACRONYMS ----------------------- %
\newacronym{dlmf}{DLMF}{Digital Library of Mathematical Functions}
\newacronym{drmf}{DRMF}{Digital Repository of Mathematical Formulae}
\newacronym{cas}{CAS}{Computer Algebra System}
\newacronym{mlp}{MLP}{Mathematical Language Parser}
\newacronym{mlp-pt}{PT}{MLP-Parse Tree}
\newacronym{bnf}{BNF}{Backus-Naur Form}
\newacronym{p2c}{P2C}{Presentation-To-Computation}
\newacronym{nlp}{NLP}{Natural Language Processing}
\newacronym{pom}{POM}{Part-of-Math}
\newacronym{csv}{CSV}{Comma-Separated Values}
\newacronym{json}{JSON}{JavaScript Object Notation}
\newacronym{teo}{TEO}{Translated Expression Object}
\newacronym{pol}{NPN}{Normal Polish Notation}
\newacronym{rpol}{RPN}{Reverse Polish Notation}
\newacronym{dag}{DAG}{Directed Acyclic Graph}
\newacronym{api}{API}{Application Programming Interface}
\newacronym{mathml}{MathML}{Mathematical Markup Language}
\newacronym{oop}{OOP}{Object-Oriented Programming}
\newacronym{cdf}{CDF}{Computable Document Format}
\newacronym{mathML}{MathML}{Mathematical Markup Language}
\newacronym{cfsf}{CFSF}{Continued Fractions for Special Functions}
\newacronym{ecf}{eCF}{Encoding Continued Fraction Knowledge}
\newacronym{nist}{NIST}{National Institute of Standards and Technology}
\newacronym{vmext}{VMEXT}{Visualizing Mathematical Expression Trees}
\newacronym{opsf}{OPSF}{Orthogonal Polynomials and Special Functions}
\newacronym{wp}{WP}{Word Processor}
\newacronym{stem}{STEM}{Science, Technology, Engineering and Mathematics}
\newacronym{gui}{GUI}{Graphical User Interface}
\newacronym{omdoc}{OMDoc}{Open Mathematical Documents}
%\makeglossaries

% ----------------------- DEFINE Simple Macros ----------------------- %
\DeclareRobustCommand{\cpi}{{\pi}}
\DeclareRobustCommand{\expe}{{e}}
\DeclareRobustCommand{\EulerConstant}{{\gamma}}
\DeclareRobustCommand{\iunit}{{i}}
\DeclareRobustCommand{\Complex}{\mathbb{C}}
\DeclareRobustCommand{\Real}{\mathbb{R}}

\newcommand{\Gudermannian}[1]{\mathrm{gd}\!\left(#1\right)}
\newcommand{\atan}[1]{\mathrm{arctan}\!\left(#1\right)}
\newcommand{\acot}[1]{\mathrm{arccot}\!\left(#1\right)}
\newcommand{\ph}[1]{\mathrm{ph}\!\left(#1\right)}


% --------------------- TODO BEFORE RELEASE --------------------- %
%
% other margins? https://www.sharelatex.com/blog/2013/08/06/thesis-series-pt2.html
%
% --------------------- TODO BEFORE RELEASE --------------------- %